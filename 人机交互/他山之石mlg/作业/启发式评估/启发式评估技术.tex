\section{启发式评估技术}

\subsection{方法}
本报告采用启发式评估可用性的方法来总结评估结果。根据知名可用性专家 Jakob
Nielsen 的说法,“启发式评估是指让一小批评估人员检查界面,并判断其是否符合公认的可用性原则(启发式)”(《如何进行启发式评估》)。(如何进行启发式评估。这种方法通常被称为“折扣”可用性技术(Nielsen, 1993, p.160),它允许评估者在一个下午的时间内发现产品或应用程序中可能存在的可用性问题。之后,通过启发式评估所发现的可用性问题可以通过更昂贵、更广泛的用户测试来研究。在进行启发式评估时,评估人员在尝试完成实际的系统任务时,会将一组预先确定的特定可用性原则与产品或网站界面进行比较。评估人员可以单独工作,稍后再汇总评估结果;也可以同时进行评估,每个人专注于几种不同的启发式评估。

\subsection{具体项目目标}
在这个项目中,小组四位成员使用十个启发式方法来发现刘洋组软工三项目界面中的可用性问题。这次评估的重点是项目界面的核心功能:情感评估工具的打分和相关图表的展示。评估人员各自准备了启发式方法的结果,然后四人在联合产品演练中一起评估了其余的启发式方法。较大的问题领域和严重程度是通过小组共识达成的。在编写本报告时没有使用 CUE 启发式评估工具。

\subsection{使用的启发式方法}
使用的十种启发式评估方法及概述如下表。

\begin{spacing}{1.3}
    \centering
    \begin{longtable}{|W{c}{0.8cm}|m{3cm}<{\centering}|m{10.2cm}|}
        \caption{启发式评估方法} \\
        \hline
        \textbf{No.} & \textbf{方法名}        & \multicolumn{1}{c|}{\textbf{描述}}                                                                                              \\ \hline
        1 & 系统状态可见 & 又称为可视性原则,即让用户知道系统在做什么;系统状态有反馈,等待时间要合理 \\ \hline
        2 & 系统与现实世界的匹配 & 又称为环境贴切原则,即使用用户语言,而不是开发者语言;贴近实际生活,而不是学术概念。总之,要使用用户可理解的表现方式;信息展示要自然贴切,逻辑正确,将用户认知成本降到最低 \\ \hline
        3 & 用户可控性/用户自由 & 又称为撤销重做原则,即操作失误可退回;用户经常会误触系统功能,这时就需要一个清晰的“紧急出口”来离开非预期状态,而不是必须拓展一个新窗口 \\ \hline
        4 & 统一和标准 & 又称为一致性原则,即同一事物和同类操作的表示要各处保持一致;不要让用户去考虑不同的单词、场景、动作是否意味着同样的东西;遵循平台规范; \\ \hline
        5 & 防错性 & 又称为防错原则,即比错误提示更友好的是——使用一种谨慎的设计方式,从一开始就防止问题的发生;要么消除容易出错的条件,要么检查这些条件,在用户触发操作时向他们提供确认选项,及早消除误操作  \\ \hline
        6 & 识别胜于回忆 & 又称为易取原则,即让用户辨认或者说识别,是一种比让用户回忆更好的方式;通过将对象 /操作 /选项可视化,减轻用户记忆负担;不应该让用户必须记住对话框的每部分信息,应该在适当的时候,系统自动提供可视化或容易检索的信息提示 \\ \hline
        7 & 使用起来灵活高效 & 又称为灵活高效原则,即通过合理的设计让用户在操作过程中更加灵活、高效;为新手和专家设计定制化的操作方式,例如:为新手提供操作引导,为专家用户提供的快捷操作,这样系统就可以同时满足有经验的用户和没有经验的用户;
        用户可体定制经常使用的操作 \\ \hline
        8 & 美观简洁 & 又称为易读性原则或易扫原则,即减少无关信息,体现简洁美感,网络用户浏览的动作更准确的形容应该是“扫”;对话框不应该包含不相关或不常用的信息;对话框中每增加一个额外的信息单元,都会与相关的信息单元争夺用户注意力,并且会降低信息的相对可见性 \\ \hline
        9 & 帮助用户识别、诊断和从错误中恢复 & 又称为容错原则,即系统出现错误时,要向用户明确的展示错误信息;准确指出问题,积极提供解决方法,协助用户尽快从错误状态中恢复正常 \\ \hline
        10 & 帮助和文档 & 又称为人性化帮助原则,即提供必要的帮助提示与说明文档;无需说明文档就能流畅的使用产品自然是极好的,但是一般文档也很有必要性;文档要易于搜索,关注用户任务,列出具体的执行步骤,并且不要太冗长 \\ \hline
    \end{longtable}
\end{spacing}


\subsection{确定问题的轻重缓急}
为了对启发式评估过程中得出的结果进行有效归类,我们将违反启发式的具体实例归纳为5个问题领域。为了进一步了解每个问题的影响,我们根据可用性原则估算了问题的严重程度以及解决问题的难易程度。影响问题严重性评级的因素包括:问题出现的频率、用户克服问题的难易程度以及问题的持续性——问题是一次就能解决还是每次尝试任务时都会困扰用户。这样就为发现的每个问题得出了双重评级,并以此来确定问题领域的优先次序,以便在本报告中进行介绍。下表定义了所采用的严重性和易修复性评级系统。严重性等级是根据 Jakob Nielsen(可用性问题的严重性等级)定义的。

\begin{spacing}{1.3}
    \centering
    \begin{longtable}{|W{c}{1.2cm}|m{13.5cm}|}
        \caption{严重程度排名} \\
        \hline
        \textbf{评级} & \multicolumn{1}{c|}{\textbf{定义}} \\ \hline
        0 & 违反启发式,但似乎不是可用性问题。 \\ \hline
        1 & 肤浅的可用性问题:用户很 容易解決或极少出现。除非有额外的时间,否则不需要在下一个版本中修复。 \\ \hline
        2 & 小的可用性问题:可能会更频繁地出现或更难克服。下一版本应优先解决这个问题。 \\ \hline
        3 & 主要可用性问题:经常出现且持续存在,或用户可能无法或不知道如何解决问题。需要解決的问题很重要,因此应优先解决。 \\ \hline
        4 & 可用性灾难:严重影响产品的使用,用户无法克服。必须在产品发布前解決这个问题。 \\ \hline
    \end{longtable}
\end{spacing}


\begin{spacing}{1.3}
    \centering
    \begin{longtable}{|W{c}{1.2cm}|m{13.5cm}|}
        \caption{轻松修复排名} \\
        \hline
        \textbf{评级} & \multicolumn{1}{c|}{\textbf{定义}} \\ \hline
        0 & 问题极易解决。一名团队成员即可在下一个版本解决。 \\ \hline
        1 & 问题容易解决。涉及特定界面元素,解决方案明确。 \\ \hline
        2 & 需要花费一定精力才能解决问题。涉及界面的多个方面,或需要开发团队在下次发布前实施更改,或解決方案不明确。 \\ \hline
        3 & 可用性问题很难解决。需要集中精力开发才能在下一个版本发布前完成,涉及界面的多个方面。解决方案可能不会立竿见影,或可能存在争议。 \\ \hline
    \end{longtable}
\end{spacing}