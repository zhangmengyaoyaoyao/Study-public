\section{摘要}
本报告洋细分析了用于评估第19组刘洋等组所选的以其软工三项目为基础的实践项目的启发式评估过程。评估本身是根据雅各布-尼尔森 (lakob Nielse)提供的启发式可用性评估方法进行的。这种方法包括评估人员在尝试完成一项系统任务时,将一组预先确定的可用性原则与应用程序或网站进行比较。

本项目使用了十种启发式方法,重点是其软工三项目的核心功能:进行情绪分析和展示所选APP在一段时间评论的情感热力图和情感趋势图。评估的目的是通过应用这十种启发式方法找出其软工三项目界面的主要缺陷:

\begin{spacing}{1.3}
\vspace{-0.5em}
\begin{multicols}{2}
    \begin{enumerate}[label=\arabic*.]
        \item 系统状态的可见度
        \item 系统和现实世界的吻合
        \item 用户享有控制权和自主权
        \item 一致性和标准化
        \item 避免出错
        \item 依赖识别而非记忆
        \item 使用的灵活性和高效性
        \item 审美感和最小化设计
        \item 帮助用户识别、诊断和恢复错误 
        \item 帮助和文档
    \end{enumerate}
\end{multicols}
\vspace{-0.5em}
\end{spacing}

评估中发现的可用性问题分为7个方面,并根据问题的严重程度和解决的难以程度进行排序。5个最严重和最容易解决的问题分别是:

\begin{enumerate}[label=\arabic*.]
    \item 情感分析结果需要二次展开
    \item 用户手册使用开发语言而不是用户语言
    \item 选择日期错误无法直接在选择界面更改
    \item 显示单词超出范围
    \item 页面使用无意义的初始值
\end{enumerate}

本报告没有详细讨论最后2个可用性问题是:“界面中的默认文本难以阅读”和“系统井不总是为用户提供有关正在执行的任务的足够信息”。这些问题没有详细讨论,因为评估人员把它们归类为肤浅的可用性问题,只有在有额外时间的情况下才应加以解决。